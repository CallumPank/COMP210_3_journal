% Please do not change the document class
\documentclass[11pt]{scrartcl}

% Please do not change these packages
\usepackage[hidelinks]{hyperref}
\usepackage[none]{hyphenat}
\usepackage{setspace}
\doublespace

% You may add additional packages here
\usepackage{amsmath}

% Please include a clear, concise, and descriptive title+
\title{Research Journal - Virtual And Mixed Reality}

\date{January 6, 2017}

% Please do not change the subtitle
\subtitle{COMP210 - Interfaces And Interaction}

% Please put your student number in the author field
\author{1605913}



\begin{document}

\maketitle

\section{Virtual And Mixed Reality}

The two prototypes that were tested are DREAMS and EREBUS these were both developed in unreal engine 4, DREAMS was played with a wireless XBOX one controller while EREBUS was played with Hydra controllers. The evaluation was undertaken by 18 university students all of which had experience with computers, 11 of the participants were games development students. This study may not be very good as it could have been more credible if they got participants from varying background other than those who used computers because this would allow them to pick up on mechanics quicker. Although at first the XBOX one controller was easier to use, and the Hydra was difficult to pick up by the end of the evaluation they could use them both easier. Motion sickness was a big problem for participants playing EREBUS this was because the DREAMS interface was better optimised.\cite{mentzelopoulos2015hardware}

New Heuristics proposed instead of Nielsen’s the examples of these are Heuristic Evaluation, games, learning and collaborative work. The reason that these new heuristics are becoming popular and needed is that they can longer work on emerging technology of Virtual reality (VR). An example where heuristics would be used is in the case of crisis management within a virtual environment this allows professionals to test within the environment requiring less time to prepare and can test new situations incredibly easily. Although there is a large amount of research going into virtual environments (VE) there are very few Heuristic evaluations for it apart from Sutcliffe and Gault\cite{sutcliffe2004heuristic} and Kalawsky\cite{kalawsky1999vruse} who are both heavily cited on the subject. This may be because of the difficulty of writing a detailed VR heuristic because of the broadness of Virtual Environments.
The paper looks at how a VR heuristic is used for crisis management on a training simulator. Sutcliffe and Gault\cite{sutcliffe2004heuristic} used 12 heuristics listed here (Show Table 1) and with that used it to test the VE they managed to uncover many serious errors within the virtual environment.
In the results section, one of the results recorded by Bordegoni et a\cite{bordegoni2009evaluation}l looked at a haptic-based system for a virtual manual assembly. An expert evaluated the system they used 9 out of the 12 heuristic points listed in Table 1. The expert was asked to perform two tasks, they then rated them from 1(Very Bad) to 4(Very Good). This does not seem very effective and will not gain as much information about the system as a quantitative evaluation would have done or an interview this will have collected the bare minimum to do with the system.\cite{hvannberg2012exploitation}

Mobile mixed reality is a new emerging trend. It uses information from its surrounding and digital data such as Pokémon go where a virtual image of Pokémon are projected on to the real-world environment using the phone's camera. This combination of real world and digital data means that this type of technology falls in between the balance of virtual and real. 
Within mobile augmented reality these apps can be roughly divided into two distinct categories AR browser and image recognition. In the 1990s to use AR mobile, you would need a head-mounted display. Whereas now phones the size of our hand can be used to host advanced technology that allows AR technology to be displayed. Using AR applications Jang Et Al\cite{jang2011overlapping}  bridged the gap between real world and reality and create an indoor space within the virtual space this space tracked the user’s movements and displayed their avatar on screen. This could be useful for those in the crisis management training of the Heuristics Journal. Where they could use this technology to create more advanced simulations, cutting cost and improving their workforces experience without using such heavy resources. This would also allow them to create more advanced situations with this space as they can create any type of scenario. Although reading on in this paper it is said by Olsson \& Salo\cite{olsson2011online} that although overall perceptions about this technology were positive the experiences would depend on the users and this could have a negative impact on using the Virtual environment technology combined with this approach.\cite{venta2014investigating}

Using VR users can ask themselves this question” What if you could become a bat—your arms acting as wings allowing you to fly through the night sky?” this already links towards the use of virtual environments and helping with crisis management allowing workers to take up this space virtually. This paper looked at how much they could push the use of an avatar, such as making the user become a lobster and making them control the lobster’s extra arms. The applications of this are immense such as using this technology to create mechanical suits to allow people to move easier and more efficiently. Although humans are creatures that use their mediated senses this means that using these new avatars and say new limbs, the change in appearance also changes the self of the person controlling the avatar. Such as when participants using an avatar with three limbs instead of two could hit more targets because they learned to control this new third limb.\cite{won2015homuncular}

Augmented reality is commonly referred to be a real environment that has elements of a virtual nature placed within it. A good example of this is Pokémon go. Within reality, there are many different sub-types such as Reality, Mixed reality, Virtuality, Augmented Virtuality and Mixed mobile reality. Although there is a large wealth of information on AR there has been very little research into the difference between virtual and real. For this definition to be found though there needs to be a clear distinction between these different realities, there are three main differences between real and Virtuality: real objects can exist objectively, a virtual object does not exist. Real objects can always be seen regardless of the medium used and virtual objects need specific mediators to be viewed such as Pokémon GO and a mobile phone. \cite{rosa2016re}

This paper looks at how the effects of VR are affecting real-life negatively and how It impacts the human’s real-life interactions and the social problems presented with using VR for an extended period. Using immersive VR our understanding of the natural world and our interactions within it are warped. For instance, some VR places you in the body of an avatar that will different to your body in the real world, this can make users embody themselves in this avatar in the real world such as in \cite{won2015homuncular} where users learned to use the third limb, this could pass onto the real world. Although using an avatar such that has all its limbs and being an amputee in the real world, can ease the pain that an amputee feels.\cite{madary2016real}
 


\bibliographystyle{ieeetran}
\bibliography{References}

\end{document}